\documentclass[journal,compsoc, 10pt, draftclsnofoot, onecolumn]{IEEEtran}

\usepackage{graphicx}
\usepackage{amssymb}
\usepackage{amsmath}
\usepackage{amsthm}
\usepackage{tabularx}
\usepackage{graphicx}

\newcommand{\subparagraph}{}
\usepackage{titlesec}

\usepackage{alltt}
\usepackage{float}
\usepackage{color}
\usepackage{url}

\usepackage{balance}
\usepackage[TABBOTCAP, tight]{subfigure}
\usepackage{enumitem}
\usepackage{pstricks, pst-node}

\usepackage{cite}
\usepackage{listings}
\usepackage{placeins}

\usepackage[margin=0.75in]{geometry}
\geometry{textheight=8.5in, textwidth=6in}

\renewcommand{\familydefault}{\sfdefault}

\newlength\tindent
\setlength{\tindent}{\parindent}
\setlength{\parindent}{0pt}
\renewcommand{\indent}{\hspace*{\tindent}}

\newcommand{\cred}[1]{{\color{red}#1}}
\newcommand{\cblue}[1]{{\color{blue}#1}}

\newcommand{\namesigdate}[2][6cm]{%
  \begin{tabular}{@{}p{#1}@{}}
    #2 \\[0.5\normalbaselineskip] \hrule \\[0.pt]
    {\small \vspace{-3em} \textit{Signature}} \\[0.5\normalbaselineskip] \hrule \\[0pt]
    {\small \vspace{-3em}\textit{Date}}
  \end{tabular}
}


\usepackage{hyperref}
\usepackage{geometry}

\lstset{
language=C,
basicstyle=\ttfamily,
commentstyle=\color{blue},
numberstyle=\color{red},
stringstyle=\color{orange}
}

\def\nameD{Devin Foulger}
\def\nameH{Hector Trujillo}
\def\nameB{Bryan Liauw}

\input{pygments.tex}

\hypersetup {
        colorlinks = true,
        urlcolor = black,
        linkcolor = black,
        pdfauthor = {\nameD\nameH\nameB},
        pdfkeywords = {},
        pdfsubject = {},
        pdfpagemode = UseNone
}

\titleformat{\section}
	{\normalfont\fontsize{15}{10}\bfseries}{\thesection}{1em}{}
\titleformat{\subsection}
	{\normalfont\fontsize{12}{15}\bfseries}{\thesubsection}{1em}{}
\titleformat{\subsubsection}
	{\normalfont\fontsize{12}{15}\bfseries}{\thesubsubsection}{1em}{}

\begin{document}

\title{\vspace{20em}Winterl Progress Report \\{\vspace{-1ex}\huge UniversityGear} \\
{\large \today}}
\author{\vspace{10ex}Devin Foulger \\{\vspace{-1ex}Hector Trujillo}
\\{\vspace{-1ex}Bryan Liauw}}

\begin{titlepage}

\maketitle
\thispagestyle{empty}

\begin{abstract}
This document will recap what we have done over the term and will describe 
what we plan to do in the following year. It will go into detail about the goals 
we have set and how we plan to accomplish them. It will also include a summary 
of the project status as it is now. It will also include a retrospective of what 
has happened throughout the term.
\end{abstract}

\end{titlepage}

\tableofcontents

\section{Introduction}
Throughout the term, we have been able to accomplish and learn a lot. We were 
able to design an application that will help people purchase college merchandise. 
We were able to come up with requirements and analyze the situation in a 
meaningful way. The team also analyzed many different technologies that we 
will be using to complete the project. 

\section{Project goals and recap}
The goal of the project is to build an Android application that will allow users 
to purchase college related merchandise. This means that users will be able to 
search for many items depending a lot of user provided information. They can 
search for specific schools and their colors as well as condition of item, or 
the price. We also set out to accomplish how to create documents in which we 
could effectively describe our project. \newline

Here our recaps from each term for each week.

\subsection{Fall Term}
\textbf{Week 2}\newline
During week 2, our group had just been in contact with our client, Luther. We 
discussed what was expected of the project and what eBay wanted. With this 
knowledge, we began work on our problem statement. This document outlined 
what was to be expected from the application that we would be developing. The 
team also set up a GitHub repo to hold all our documents and code. \newline

\textbf{Week 3}\newline
We completed our problem statement during week 3. After many emails with our 
client, we came to an agreement with the problem statement.  \newline

\textbf{Week 4}\newline
During week 4, we had began to make plans on what needed to be revised in the 
problem statement. The team had also created developer accounts on the eBay 
website and began looking at the APIs we would be using. We had also received 
a small rundown on what requirements will need to be met. Finally, we also began 
the weekly meetings with our TA.\newline

\textbf{Week 5}\newline
This week, we worked towards completing our requirements document. We had 
figured out what would need to be accomplished. Our client had also provided us 
with a mock up of what our application could look like. This helped with the 
creation of what requirements must be met.\newline

\textbf{Week 6}\newline
Over the course of week 6, the team accomplished a lot. We finished our 
requirements document and met with our client in Portland. After meeting with 
the team at eBay, it really felt like they wanted us to succeed. They helped us 
to better define the requirements that needed to be completed.\newline

\textbf{Week 7}\newline
During week 7, we began work on the tech review. We had split the requirements 
into three major parts: Searching and gathering data, UI and presenting data, and 
purchasing the items. After that, each team member focused on the parts that they 
had been assigned. \newline

\textbf{Week 8}\newline
Over week 8, the team did a mock design of what the application would look like. 
This was the beginning of the design document. Each member had also completed 
their part of the tech review. The review featured many different technologies that 
we would be using, discussing the benefits of each one. \newline

\textbf{Week 9}\newline
Large portions of the design document were completed over week 9. The group 
created a UML to demonstrate what classes might be used during development. 
\newline

\textbf{Week 10}\newline
The final week of the term, we completed our design document. However, we still 
needed to begin work on the fall progress report. \newline

Overall, we set out to accomplish many goals over the term. The end result of 
each of our documents provides us with a good start for development. 

\subsection{Winter Term}
\textbf{Week 1}\newline
During week 1, we talked about when we were going to meet to do some paired 
programming. We also uploaded our blank Android project to GitHub to get 
started. \newline

\textbf{Week 2}\newline
This week we accomplished creating our search activity, however the class does 
not currently work with eBay's APIs. \newline

\textbf{Week 3}\newline
For this week, we were able to get eBay's search API integrated with our 
search class allowing for us to make searches. On top of that, we also started 
the list view activity and the item class. The item class will allow us to create 
items in any class. The list view allows us to view the items by using a recycler 
view.\newline

\textbf{Week 4}\newline
During week 4, we have been connecting everyones pieces. This means that we 
can now go from searching to the list view correctly displaying search results. 
The list view correctly displays the title, image, and price of the item. The 
item class is also continually being added to, depending on what new variables
we will be needing. \newline

\textbf{Week 5}\newline
This week, we have been updating the search activity to apply filters. It will 
also now only select BIN items and specific colleges. The single item view has 
also been completed. The user can now select an item from the list view to view 
in full detail in the single item view. This used eBay's API to get the correct 
information on a specific item. The item class has also been updated to handle 
many of the given fields for a single item. 


\section{Project status}
The design for our application, UniversityGear, has been fairly complete. We 
have detailed the functional requirements of our project. We have also completed 
a technology review in which we discovered what technologies would be the most 
efficient with our application. Last, we created a design document to detail how 
we were going to implement the technologies we have found into our application.
After many weeks of communication with our client, we have arrived at the 
development phase.\newline

We have developed over half of our application. So far, we have implemented the 
ability to search, view, and select single items. This alone is a large part of 
the project. In the coming weeks, users will be able to purchase items, as well 
as use several filters to find the items that they need. These classes we have 
created also interact with eBay's APIs

\section{Impediments and solutions}
Over the course of the term, our team managed to avoid many impediments. However, 
we did need to revise our problem statement. In order to solve this, we re-wrote the 
problem statement with the revisions provided by the professors of the capstone 
class. 

Another problem is the design documents. The format that is required is not something we are familiar with. There is a lot of questions regarding what to put where. After looking at several examples, as well as asking people, we managed to get it done in the end.

\section{Retrospective}

\begin{table}[!h]
\centering
\caption{Retrospectives}
\label{my-label}
\begin{tabularx}{\textwidth}{X|X|X}
\hline
\textbf{Positives} & \textbf{Deltas} & \textbf{Actions} \\ \hline
Completed the problem statement                  &  Didn't meet the criteria                &Rewrote the report to define the problem better                 \\ \hline
Met with the client in Portland                  &     Could meet with clients more frequently            & Arrange more meetings on-line                 \\ \hline
Completed the requirements document                  & Not quantifying goals, spelling errors, and missing an abstract                & Proofreading it more and asking for more clarifications prior to submitting it                 \\ \hline
Completed the technology review                 & Writing could be improved                & Go to the Writing Center in order to see missed mistakes                 \\ \hline
Completed the design document                  & Can start earlier                 &Talk more to TA and instructor on clarifying some matters                  \\ \hline
\end{tabularx}
\end{table}

\FloatBarrier
\section{Conclusion}
During the course of the term we have learned many things. This includes items 
from different IEEE formats used in the real world to best practices when gathering 
project requirements. We also learned how to look at projects from the 10,000 
foot level instead of trying to focus on specific details. We will begin development 
on our project during winter term. However, we are hoping to make progress 
during winter break. 

\end{document}
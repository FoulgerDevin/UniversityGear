\documentclass[letterpaper, 10pt, draftclsnofoot, onecolumn]{IEEEtran}

\usepackage{graphicx}
\usepackage{amssymb}
\usepackage{amsmath}
\usepackage{amsthm}

\usepackage{alltt}
\usepackage{float}
\usepackage{color}
\usepackage{url}

\usepackage{balance}
\usepackage[TABBOTCAP, tight]{subfigure}
\usepackage{enumitem}
\usepackage{pstricks, pst-node}

\usepackage{cite}
\usepackage{listings}

\usepackage[margin=0.75in]{geometry}
\geometry{textheight=8.5in, textwidth=6in}

\newcommand{\cred}[1]{{\color{red}#1}}
\newcommand{\cblue}[1]{{\color{blue}#1}}

\newcommand{\namesigdate}[2][6cm]{%
  \begin{tabular}{@{}p{#1}@{}}
    #2 \\[0.5\normalbaselineskip] \hrule \\[0pt]
    {\small \textit{Signature}} \\[0.5\normalbaselineskip] \hrule \\[0pt]
    {\small \textit{Date}}
  \end{tabular}
}

\usepackage{hyperref}
\usepackage{geometry}

\def\name{Devin Foulger, Bryan Liauw, and Hector Trujillo }

\input{pygments.tex}

\hypersetup {
        colorlinks = true,
        urlcolor = black,
        pdfauthor = {\name},
        pdfkeywords = {},
        pdfsubject = {},
        pdfpagemode = UseNone
}

\pagenumbering{gobble}
\begin{document}

\title{ Problem Statement}
\author{\name\vspace{-10ex}}
\maketitle
\begin{center}
	\today\vspace{2ex}
\end{center}
\section*{Abstract}
	In order to stay competitive in the online retailer business, eBay recently 
	created multiple APIs that allow online retailers to start their business 
	easier. These APIs will be released to the public soon. So, in order to test 
	the applicability, eBay would like us to test up to three of their APIs by 
	creating an application that demonstrates the APIs use. The application will 
	consist of college merchandise and allow users to search for and purchase 
	merchandise related to any college and will target the new Android operating 
	system, N OS. The goal for eBay is that their APIs are tested and well documented 
	for later use by other third party developers and to get more eBay related apps 
	onto the Google Play store.

\section*{Problem Definition}
	EBay wants to test and demonstrate the functionality of their newly created APIs. 
	In order to test the applicability of the APIs, eBay would like us to utilize up to 
	three of their APIs to create an application. Since these APIs will be used by 
	third-party developers, eBay would like an example of how each of these is used
	 as a demonstration. The APIs will help users create targeted eBay experiences.
	 This is done by allowing developers to access eBay's catalog of items. It is 
	also imperative that we evaluate the documentation for the APIs and determine 
	if it is accurate enough for developers to integrate with their own catalog. 
	Also, they would like to know how well the app is going to work on the new 
	Android operating system. 

\section*{Proposed Solution}
	In order to test the APIs, we are going to make an Android application that 
	functions as an online retailer, similar to eBay. In order to limit the scope, 
	our application will sell only college related merchandises. The application 
	will allow users to search for college merchandise in which they will be able 
	to purchase from the Ebay catalog of items. We will also be documenting our 
	use with the APIs and any bugs that we may encounter. This document will be 
	based on the output of the API and we will  compare this to the documentation 
	provided by eBay. 
	\par At the Engineering Expo, we will demonstrate our application on an Android
	 device. The demonstration will consist with the basic functionality of the 
	application. This includes searching for college merchandise using different 
	search filters that can be changed as needed. During the demonstration, we will
	 also identify what functionality is a result of using eBay's public APIs that 
	we tested. This application will serve as a prototype for vendors on eBay to 
	develop their own apps to deliver a centralized shopping experience.

\newpage
\section*{Performance Metrics}
	There are several ways we can measure the success of our project. One, the 
	application should be live in the Google Play store. Two, would be to 
	demonstrate the clear use of eBay's APIs through our application. This is done 
	by creating an interface that visualizes the public API. The data that is 
	visualized should also be correct. Finally, eBay will be satisfied by the 
	documents that we provide to them.


\vspace{2ex}\noindent 
\namesigdate{Luther Boorn} \hfill
\namesigdate{Devin Foulger} \hfill

\vspace{4ex}\noindent
\namesigdate{Hector Trujillo} \hfill
\namesigdate{Brian Liauw} \hfill

\end{document}
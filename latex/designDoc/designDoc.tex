\documentclass[journal,compsoc, 10pt, draftclsnofoot, onecolumn]{IEEEtran}

\usepackage{graphicx}
\usepackage{amssymb}
\usepackage{amsmath}
\usepackage{amsthm}
\usepackage{tabularx}
\usepackage{graphicx}

\newcommand{\subparagraph}{}
\usepackage{titlesec}

\usepackage{alltt}
\usepackage{float}
\usepackage{color}
\usepackage{url}

\usepackage{balance}
\usepackage[TABBOTCAP, tight]{subfigure}
\usepackage{enumitem}
\usepackage{pstricks, pst-node}

\usepackage{cite}
\usepackage{listings}
\usepackage{placeins}

\usepackage[margin=0.75in]{geometry}
\geometry{textheight=8.5in, textwidth=6in}

\renewcommand{\familydefault}{\sfdefault}


\newlength\tindent
\setlength{\tindent}{\parindent}
\setlength{\parindent}{0pt}
\renewcommand{\indent}{\hspace*{\tindent}}

\newcommand{\cred}[1]{{\color{red}#1}}
\newcommand{\cblue}[1]{{\color{blue}#1}}

\newcommand{\namesigdate}[2][6cm]{%
  \begin{tabular}{@{}p{#1}@{}}
    #2 \\[0.5\normalbaselineskip] \hrule \\[0pt]
    {\small \textit{Signature}} \\[0.5\normalbaselineskip] \hrule \\[0pt]
    {\small \textit{Date}}
  \end{tabular}
}


\usepackage{hyperref}
\usepackage{geometry}

\lstset{
language=C,
basicstyle=\ttfamily,
commentstyle=\color{blue},
keywordstyle=\color{green},
numberstyle=\color{red},
stringstyle=\color{orange}
}

\def\nameD{Devin Foulger}
\def\nameH{Hector Trujillo}
\def\nameB{Bryan Liauw}

\input{pygments.tex}

\hypersetup {
        colorlinks = true,
        urlcolor = black,
        linkcolor = black,
        pdfauthor = {\nameD\nameH\nameB},
        pdfkeywords = {},
        pdfsubject = {},
        pdfpagemode = UseNone
}

\titleformat{\section}
	{\normalfont\fontsize{15}{10}\bfseries}{\thesection}{1em}{}
\titleformat{\subsection}
	{\normalfont\fontsize{12}{15}\bfseries}{\thesubsection}{1em}{}
\titleformat{\subsubsection}
	{\normalfont\fontsize{12}{15}\bfseries}{\thesubsubsection}{1em}{}

\begin{document}

\title{\vspace{20em}Design Document \\{\vspace{-1ex}\huge UniversityGear} \\
{\large \today}}
\author{\vspace{10ex}Devin Foulger \\{\vspace{-1ex}Hector Trujillo}
\\{\vspace{-1ex}Bryan Liauw}}

\begin{titlepage}

\maketitle
\thispagestyle{empty}

\end{titlepage}

\tableofcontents


\section{Introduction}
\subsection{Scope}
The application that is going to be developed will be for Android devices, 
specifically for N and M OS. It will allow users to purchase college merchandise
 via eBay's large collection of items. The user will be able to search for items
 from different colleges or by specific search criteria. The project will begin 
development in December, 2016 and end in May, 2017.

\subsection{Purpose}
The purpose of this design document is to outline how the application will be 
structured to satisfy the requirements we have set. It will go into depth about 
the design details of our software.

\subsection{Intended Audience}
The intended audience is for the team "4Credit" and their client. It is also for
 the professors of the senior capstone class at Oregon State University.

\subsection{References}
NONE YET.

\section{Definitions}

\begin{table}[!h]
\centering
\caption{Definitions}
\label{my-label}
\begin{tabularx}{\textwidth}{l|X}
\hline
\textbf{Term}               & \textbf{Definition}                                                                                                           \\ \hline
API                    	      & Application Programming Interface, allows developers to develop applications that connect to eBay's large inventory of items. \\ \hline                                                           \\ \hline
Vendors               	      & Individuals or companies that sell or will sell items on eBay                                                                 \\ \hline
Filters                	      & Search criteria that will allow the user to more easily find the item that they are looking for.                              \\ \hline
Third-Party Developers & Individuals outside of eBay that will be using eBay's public APIs to create their own applications.                           \\ \hline
Application            	      & An Android application that uses eBay's public APIs.                                                                          \\ \hline
College Merchandise     & Items such as school supplies for Higher Ed, fan gear, etc...                                                                 \\ \hline
User                   	      & Any one who will be using the Android Application utilizing eBay's APIs.                                                      \\ \hline
Client                 	      & eBay                                                                                                                          \\ \hline
BIN                    	      & Buy It Now                                                                                                                    \\ \hline
N                   	      & Android's new Nougat operating system 
\\ \hline
M                   	      & Android's Marshmallow operating system                                                                                                                   
\\ \hline
MVC                   	      & A design pattern called Model View Controller
\\ \hline
\end{tabularx}
\end{table}
\FloatBarrier


\section{Conceptual model for software design descriptions}
The basic concepts and context of the design document will be explained in this 
section of the document.

\subsection{Software design in context}
An object oriented approach will be taken and we will be using a specific 
design pattern called MVC. This will allow us to maintain some form of a 
separation of concerns. In turn, it will make development much smoother. We will
 also have to keep track of an API layer, as we will be using eBay's new APIs. 
It is imperative that our application also works on the M and N Android 
operating systems.

\subsection{Software design descriptions within the life cycle}
\subsubsection{Influences on design document preparation}
The software requirements specification (SRS) plays a large role in influencing 
the design document. This is because the SRS defines all of the functional and 
interface requirements for the successful completion of the project.

\subsubsection{Design verification and design role in validation}
Test cases will be developed simultaneously to development phase. This will 
allow us to validate that our application is successfully working as expected. 
The last step of validation we will complete if our application successfully 
met the requirements defined in the SRS.


\section{Design description information content}
This section of the design document will identify how the application will be 
implemented and designed. 

\subsection{Design document identification}

\subsection{Design stakeholders and their concerns}
The stakeholders of the project our the developers and their clients. The 
biggest concern is that the product will meet all of the requirements specified 
in the SRS. The application must also contain design that is described in this 
document. 

\subsection{Design views}
We will be using different processes for representing diagrams. We will be using
 UML for describing our functions and classes.

\subsection{Design viewpoints}
The design viewpoints will be described with a combination of UML diagrams and 
short descriptions. The descriptions could include information regarding the 
context, composition, logical, or interactive viewpoints. Each viewpoint will 
also have a specific name.

\subsection{Design elements}
There are a lot of design attributes we will be discussing in our document. Our 
design entities will consist of the APIs and classes that we will be using to 
accomplish our goals. Each entity will have a name, a type, and a purpose. The 
name attribute is simply the name of the element or entity. The type attributes 
will consist of what kind of class or API we will be using. The purpose attribute 
is to give a description as to why an element might exist. Last, an author 
attribute will be used to identify which designer will be responsible for the 
element.

\subsection{Design rationale}
It is important that our application performs well on the N and M Android operating 
systems. We will make sure that the application is also easily maintainable 
for future developers. We will also provide documentation of the design process 
so that other developers will understand the structure of the application.


\section{Design viewpoints}
The main design viewpoints will be explained here. The exact viewpoints are the 
context, composition, logical, dependency, interfaction, and algorithm viewpoints.

\subsection{Context viewpoint}
The context viewpoint will document the functionality between the user and the 
application. There are two main functionalities the user should be using: 
searching and purchasing merchandise. This sections serves to define the user 
interaction with the app. It will also talk about the sub functions that will 
be used to facilitate the main functions. \newline

The user of the application will also have to perform other smaller actions in 
order to get the full use. For example, they will have to download the application 
from the Google Play store. From there, they would be required to search for 
something in order to purchase it. They will be searching for specific schools, 
prices, condition of item, and other search criteria. They will then be presented 
with a list of items in which they will be able to select and view that item in 
greater detail. They will also be able to purchase the item and then they will 
need to enter in their name and shipping information.

\subsection{Composition viewpoint}
The application will be divided into three different parts. The first being the 
search feature, which will allow the user to find a specific item. The second 
will be the purchasing feature, which will allow the user to purchase any item 
they have found. Last, the user should be able to view in detail everything 
about the item they have searched for. All these pieces need to be connected 
together and coherently function as one piece. 

\begin{figure}[h]
\centering
\caption{UML Diagram for the for main classes}
\includegraphics[scale=.65]{projectUML}
\end{figure}
\FloatBarrier

\subsubsection*{Function attribute}
The search class will allow users to search for their items. This class will need 
several variables, which will be passed into a function that calls an API. The 
output will be a list of items that are related to the search criteria provided 
by the user. The storage class will allow the user to save their search filers 
for later use so they don't have to keep entering the same data each time. This 
requires the same informatuion priovded to the search. From there, the class will 
either save, load, or delete this data. The item class will be used to hold and 
send the specific item data to the purchase class. The purchase class will allow 
the user to purchase the item that they have selected. It will also send a 
confirmation if the user's purchase has been completed.

\subsection{Logical viewpoint}
Our application will consist of many classes, but they will either be a model or 
a controller. These classes will be the Search class, Purchase class, Item class, 
and the Storage class.

\subsubsection*{The search class}
This class will allow users to search for items that they would like to view or 
purchsae. It will share a relationship with the item and storage classes. It 
will have a method for searching that needs the search terms that have been 
provided by the user. 

\begin{figure}[h]
\centering
\caption{UML for the search class}
\includegraphics[scale=.9]{searchUML}
\end{figure}
\FloatBarrier

\subsubsection*{The storage class}
This class will be used to save the users search criteria. This is so the user 
will not have to enter in the same information again, considering that they 
will most likely search for the same schools over and over again. It will have 
the ability to save, load, and delete filters. It will also need the information 
provded by the user such as the school and searchTerms.

\begin{figure}[h]
\centering
\caption{UML for the storage class}
\includegraphics[scale=.9]{storageUML}
\end{figure}
\FloatBarrier

\subsubsection*{The item class}
The item class will hold all the information on a signle item. It will be 
displayed to the user in either a list view or a single item view. The information 
will also be sent the purchase class so that the user may be able to purchase 
the item.

\begin{figure}[h]
\centering
\caption{UML for the item class}
\includegraphics[scale=.9]{itemUML}
\end{figure}
\FloatBarrier

\subsubsection*{The purchase class}
The purchase class will allow the user to purchase the item that they have 
selected. It will need the item key when purchasing an item. It will also notify
the user in some way that they know the purchase has been completed.

\begin{figure}[h]
\centering
\caption{UML for the purchase class}
\includegraphics[scale=.9]{purchaseUML}
\end{figure}
\FloatBarrier

\subsection{Dependency viewpoint}
This section will describe the relationship between each of the classes and how 
they depend on each other. 

\subsection{Interaction viewpoint}

\subsection{Algorithm viewpoint}
The algorithms viewpoints will describe the different algorithms that we will be
 using to correct users search mistakes. The Levenshtein Distance algorithm that 
we are going to use is going to be attached to the search bar that is in most 
pages of the UI. The algorithm will run every time the users finish typing and 
submitting their search. The algorithm will run regardless whether the keyword 
needs to be corrected or not. The fix will happen if the user has written 
something that is not quite correct; that is, the distance between the intended 
string and the string that user has written does not have a 0 value. Once this 
case happens, we will find the keyword with the minimum distance from the string 
user has written and replace user's keyword with that.

\subsubsection*{Design concerns}

\subsubsection*{Design elements}

\subsubsection*{Processing attribute}

\subsubsection*{Examples}

\end{document}